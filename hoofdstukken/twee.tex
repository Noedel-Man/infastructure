\section{Application Layer}
Dit hoofdstuk gaat over de Application Layer. waar eigenlijk al je netwerk programma's opdraaien.
\subsection{Principles of Network Applications}
Het principe van netwerk layers is dat je je maar meestal zorgen hoeft te maken over 1 of 2 lagen van het netwerk systeem. Wanneer je een applicatie schijft voor een eind-systeem hoef je dus alleen maar zorgen te maken over hoe de applicatie layer jou programma implementeert.

\subsubsection{Network Application Architectures}
Er zijn twee verschillende application netwerk architecturen; \textbf{P2P} (peer-to-peer) en \textbf{client-server} architectuur. Je bent niet gelimiteerd aan deze twee architecturen maar het zijn de enige die op dit moment bestaan. (je kan zelf iets nieuws verzinnen als je genoeg vrije tijd hebt).

\paragraph{client-server}
Een client-server is de meest simpele en meest traditionele architectuur. Het gebruikt twee system een \textit{client} en \textit{server} De server moet op een statische plek staan en 24/7 beschikbaar zijn. De client daarin tegen kan dynamiche zijn (van IP verwisselen) en uit en aan gaan wanneer gewenst. De server neemt geen dienst af van de client.
\paragraph{Peer-to-peer}
Een peer-to-peer verbinding word onder anderen gebruikt bij Torrents en video gesprekken. De verbinding word onderling gedaan zonder centrale server. Dit systeem schaalt ook veel beter omdat wanneer iedereen een bestand wilt downloaden het ook door meer mensen word geupload.

\subsubsection{Processes Communicating}
Wanneer een \textbf{process} wilt communiceren met een ander process op het Internet heeft het 2 dingen nodig, het adres van de het anderen end-systeem en 
\textbf{Sockits} of \textbf{ports} zij



\subsection{The Web and HTTP}