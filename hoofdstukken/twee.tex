\section{Application Layer}
Dit hoofdstuk gaat over de Application Layer. waar eigenlijk al je netwerk programma's opdraaien.
\subsection{Principles of Network Applications}
Het principe van netwerk layers is dat je je maar meestal zorgen hoeft te maken over 1 of 2 lagen van het netwerk systeem. Wanneer je een applicatie schijft voor een eind-systeem hoef je dus alleen maar zorgen te maken over hoe de applicatie layer jou programma implementeert.

\subsubsection{Network Application Architectures}
Er zijn twee verschillende application netwerk architecturen; \textbf{P2P} (peer-to-peer) en \textbf{client-server} architectuur. Je bent niet gelimiteerd aan deze twee architecturen maar het zijn de enige die op dit moment bestaan. (je kan zelf iets nieuws verzinnen als je genoeg vrije tijd hebt).

\subsubsection*{client-server}
Een client-server is de meest simpele en meest traditionele architectuur. Het gebruikt twee system een \textit{client} en \textit{server} De server moet op een statische plek staan en 24/7 beschikbaar zijn. De client daarin tegen kan dynamiche zijn (van IP verwisselen) en uit en aan gaan wanneer gewenst. De server neemt geen dienst af van de client.
\subsubsection*{Peer-to-peer}
Een peer-to-peer verbinding word onder anderen gebruikt bij Torrents en video gesprekken. De verbinding word onderling gedaan zonder centrale server. Dit systeem schaalt ook veel beter omdat wanneer iedereen een bestand wilt downloaden het ook door meer mensen word geupload.

\subsubsection{Processes Communicating}
Wanneer een \textbf{process} wilt communiceren met een ander process op het Internet heeft het 2 dingen nodig, het adres van de het anderen end-systeem en 
\textbf{Sockits} of \textbf{ports} zij



\subsection{The Web and HTTP}
Tot 1990 werd het internet grotendeels door onderzoekers en scholieren gebruikt. Weinig mensen wisten er nog van. Later in 1994 werd de eerste webbrowser ontwikkeld en daardoor werd het internet onder de het normale volk ook ontamd.


\subsubsection{Overview of HTTP}
\textbf{Hyper Text Transer Protocol (HTTP)} word gebruikt om documenten (objecten) over het internet te versturen. Wanneer een object (index.html) verwijst naar een ander object (favicon.ico) word deze ook opgehaald. HTTP maakt gebruik van TCP en word beschouwd als een \textbf{stateless protocol} dat houd in dat de server geen data bijhoud van haar clients. Wanneer een client 4x hetzelfde bestand zou aanvragen zal de http server de 4e aanvraag hetzelfde behandelen als de eerste.

\subsubsection{Non-Persistent and Persistent Connections}

HTTP kent twee soorten verbindingen: Non-Persistent en Persistent.
\subsubsection*{Non-Persistent}
Wanneer een client een object aanvraagt van een HTTP server gaan ze eerste een \textbf{three-way-handshake} doen. De client vraagt aan de server om een TCP verbinden optestellen en de server antwoord vervolgends met \textit{OK}. Daarna vraagt de client een object aan (bijv. \textit{/mydir/index.html}) De server antwoord dan met het bestand \textit{index.html} en de verbinden word verbroken.
\newline

De client leest het bestand en ziet dat hij nog 10 plaatjes moet laden. Hij gaat dan weer een three-way-handshake doen en het hele verhaal begint weer opnieuw.

\subsubsection*{Round-Trip Time (RTT)}
RTT is het het tijd wat het duurt voor een pakketje heen en terug te sturen van client naar een server. Propegation delay heen en terug.
In het vorige voorbeeld zijn er twee RTT. De eerste is wanneer de client connectie aanvraagt aan de server en antwoord krijgt. De twee is het bestand wat hij vraagt en ook krijgt van de server. Voor elke extra object krijg je bij een non-persisten verbinding \textbf{2} RTT's (een om verbinden vast te leggen, en een om het bestand te krijgen.)

\subsubsection*{Persistent}
Een persistent verbinden is veel zuiniger met data dan een Non-persistent verbinden. Want het houd de TCP connectie open tot een bepaalde timeout tijd. Dus in ons voorbeeld met 1 html pagina en 10 plaatjes heb je dus maar 12 RTT's op een persistent verbinding 
(1 RTT voor TCP verbinden + html bestand + (plaatje * 10)) In plaatjs van 11 * 2 = 22 RTT's bij een Non-persistent verbinding.

\subsubsection{HTTP Message Format}
Er zijn twee verschillende HTTP bericht formaten, een request en een response. Hieronder volgt een voorbeeld van een request:
\newline
\texttt{GET /tools/index.html HTTP/1.1\newline
		Host: 			www.noeel.nl\newline
		Connection: 	close\newline
		User-agent: 	Mozilla/5.0\newline
		Accept-language: nl\newline
	}

De request is geschreven in ASCII. De eerste zin is de \textbf{request line} de lijnen die daarop volgen zijn de \textbf{header lines}. Daarna komt er een wit-regel gevolgt door de \textbf{Entity body} maar die is nu leeg omdat het een request is en geen response of \texttt{POST}
\newline
\subsubsection*{Request line}
De request kan bestaan uit een \texttt{GET, POST, HEAD, PUT} of \texttt{DELETE}

\texttt{GET} request een bestand, \texttt{POST} stuur data naar een server (de content staat dan in de body)
\texttt{HEAD} voor debugging, vraag alleen de head van een response op. \texttt{PUT} plaats een bestand en \texttt{DELETE} verwijder een bestand.
Het word daarna gevolgt door de locatie en de HTTP versie.

\subsubsection*{Header lines}
\texttt{Host:} verklaard de host waar de request naar toe moet gaan, is nodig voor proxy's
\texttt{Connection:} of het een persistant (\texttt{open}) of non-persistant (\texttt{close}).
Data zoals \texttt{User-agent} en \texttt{Accept-language} worden gebruikt om verschillende data weer te geven per taal/browser.

\subsubsection*{HTTP Response}
\texttt{HTTP/1.1 200 OK \newline
		Connection: close \newline
		Date: Tue, 18 Aug 2015 15:44:04 GMT \newline
		Server: Apache/2.2.3 (CentOS) \newline
		Last-Modified: Tue, 18 Aug 2015 05:44:04 GMT \newline
		Content-Lenght: 6532 \newline
		Content-Type: text/html \newline \newline
		data data data .... }
\newline
In de Status line staat de HTTP versie gevolgt door de statis code en statis zin. (200 betekent dat alles goed is) \texttt{Connection: close} betekent dat de server de TCP verbinding zal sluiten hierna. \texttt{Date} De datum en tijd dat de server het bestand van het file system heeft gekregen. \texttt{Server:} Allemaal server data, welke http server hij draaid, welke versie en welk OS. \texttt{Last-Modified:} handig voor caching enzo. \texttt{Content-Lenght} hoelang is de data en \texttt{Content-Type} wat voor soort data het is.

\subsubsection{User-Server Interaction: Cookies}
Omdat HTTP stateless is weet de server niet of de client vaker gebruikt maakt van zijn diensten. Daarom bestaan er Cookies, een Cookie is een string die de server kan zetten met de HTTP header: \texttt{Set-cookie: myCookieValue} Wanneer de client een nieuwe request maakt naar dezelfde server stuurt hij altijd in zijn HTTP header (\texttt{Cookie: myCookieValue}) zodat de server dit kan processen en client speceviek diensten aan kan bieden.


\subsubsection{Web Caching}
Je kan ook een caching server hebben. Dit is een server dat een kopie maakt van een bestaande server en dit op een andre plek host. Dit kan handig zijn als de normale server een hoge ping heeft en dus een lage response tijd heeft. Om de caching server te berijken moet je zelf in je browser (of andere applicatie) instellen dat je met de caching server verbinding wilt maken inplaats van de bestaande server.
\newline
Wanneer je een HTTP GET request stuurt naar een caching server kijkt hij of hij het bestand heeft en stuurt deze dan naar jou toe. Wanneer het bestand ontbreekt vraagt hij het zelf op bij de orginele server en stuurt het dan naar jou toe en slaat het ook meteen op.
\subsubsection*{Conditional GET}
Het grote nadeel van een caching server is dat hij een out-of-date bestand kan hosten. Om dat te voorkomen checked een caching server eens in de paar weken of er geen nieuwer bestand bestaat. Dat gaat omgeveer zo:
/newline
Eerst vraagt de caching server het orginele bestand aan \newline
\texttt{GET /img/logo.png HTTP/1.1 \newline
		Host: www.mylogoImages.com \newline \newline
	}
De orginele server geeft een respone: \newline
\texttt{HTTP/1.1 200 OK \newline
		Date: Sat, 3 Oct 2015 15:39:29 \newline
		Server: Apache/1.3.0 (Unix) \newline
		Last-Modified: Wed, 9 Sep 2015 09:23:24 \newline
		Content-Type: image/png \newline \newline
		data data data ... \newline}
Wanneer er een bepaalde tijd voorbij is en de caching server is bang dat zijn bestand out-of-date is checkt hij bij de server of er een nieuwere versie beschikbaar is: \newline
\texttt{GET /img/logo.png HTTP/1.1 \newline
	Host: www.mylogoImages.com \newline
	If-modified-since: Wed, 9 Sep 2015 09:23:24 \newline \newline
}
Wanneer er geen nieuwe versie is beantwoord de server met een $304$: \newline
\texttt{HTTP/1.1 304 Not Modified \newline
	Date: Sat, 10 Oct 2015 15:39:29 \newline
	Server: Apache/1.3.0 (Unix) \newline}
	
\subsection{Electronic Mail in the Internet}
E-mail bestaat uit drie delen: \textbf{user agents} (het client programma op het eind-systeem van de gebruiker), \textbf{mail servers} (de servers die mails ontvangen en versturen) deze hebben zowel een client als server rol en het protocol om mails te versturen \textbf{Simple Mail Transfer Protocol (SMTP)}
\subsubsection{SMTP}
//TODO
\subsubsection{Comparison with HTTP}

\subsubsection{Mail Message Formats}

\subsubsection{Mail Access Protocols}

\subsection{DNS-The Internet's Directory Service}
Servers hebben vaak twee adresssen waarmee je met ze kan verbinden: hostname \textit{www.noeel.nl} en het ip adress \textit{185.182.56.20}. Computers willen met het ip addres verbinden omdat dat ook echt iets betekent. Maar voor mensen is dat moeilijk te onthouden. Daarom bestaan er hostnamen die verwijzen naar het ip-adres zodat je die in kan typen in je webbrowser.
\subsubsection{Services Provided by DNS}
Wanneer jij verbinding wil maken met een hostname verstuurd je computer een verzoek naar een DNS server om het ip addres op te zoeken van de hostname. De DNS server geeft dan het ipaddres en je kan verbinding maken.
\newline
Een DNS server bied een aantal services:
\paragraph{Host aliasing}
Een sub domain (\textit{sub1.sub2.domain.org}) word een \textbf{canonical hostname} genoemd deze verwijzen naar \textit{www.domain.org} of \textit{domain.org}
\paragraph{Mail server aliasing}
Je kan van elk domain naam ook een MX-record opvragen, dit is het addres van de mail server.
\paragraph{load distribution}
Je kan ook meerdere records van een soort (MX,A etc.) opgeven bij een DNS server. Hij kiest er dan per request een random uit zodat niet een server van je word belast.

\subsubsection{Overview of How DNS Works}
Er bestaan 4 verschillende DNS servers:
\paragraph{Root DNS servers}
Dit zijn servers die alleen een lijst hebben met TLD DNS servers. Wanneer jij een request maakt naar \textit{www.google.nl} verwijst deze server je door naar een NL TLD DNS server
\paragraph{Top-Level domain server}
elk top-level domein naam (nl, com, org, de etc.) heeft haar eigen TLD DNS server. In deze server staan alle \textbf{Authoritative DNS servers} van elk domain naam onder dat TLD. Wanneer je een request doet voor het ip van google.nl geeft hij je de Authoritative DNS servers van google.nl.
\paragraph{Authoritative DNS servers}
Dit zijn de DNS servers die ook daatwerkelijk het antwoord voor je hebben. deze hebben alle records van een domain naam en geven je ook het ipaddres van je bestemming.
\paragraph{local DNS server}
Dit zijn vaak DNS servers van je ISP (Ziggo bijv.) Deze hebben een lijst van alle Root DNS servers en vragen aan hen het ip addres van je request. Hier maak jij dus ook verbinding mee.
\newline
\subsubsection*{DNS request}
Wanneer jij een vraag naar een DNS server doet voor het ophalen van een ip address kan dat op twee verschillende manier gaan.
\paragraph{Various DNS servers}
\begin{itemize}
	\item Een eind-systeem maakt doet een DNS request voor \textit{www.noeel.nl} naar een locale DNS server
	\item locale server doet een request naar de root server
	\item de root server leest het \textit{.nl} gedeelte en returnt een adres van een NL TDL DNS server
	\item de locale server ontvangt de request en stuurt een DNS request voor \textit{www.noeel.nl} naar de NL TDL DNS server\
	\item de TLD server leest het bericht en returnt de authoritative DNS server voor noeel.nl
	\item de local server stuurt een bericht naar de authoritative DNS server en krijgt de record die hij verwachte
	\item de local server stuurt het ip-adres door naar het eind-systeem dat het eerder aanvroeg.
	\item profit!
\end{itemize}
\paragraph{Recursive queries in DNS}
Dit werkt op een anderen manier als de varios DNS server variant. inplaats van dat de server steeds een adres returned van een ander DNS server gaat hij zelf vragen aan die server vor het adres. Maar dit komt minder vaak voor.
\newline
\newline
DNS servers cashen ook al hun resultaten om verkeer te verminderen, en om sneller te zijn. Meestal word iets voor 2 dagen gecached. Maar als web-host kan je dit ook korter of langer instellen.
\subsubsection{DNS Records and Messages}
\subsection{Peer-to-Peer File Distribution}
