\section{Transport Layer}
\subsection{introduction and Transport-Layer Services}
Een transport layer protocol zorgt voor logiche communicatie tussen applicaties van verschillende hosts.
\subsubsection{Relationship between Transport and Network Layers}
\subsubsection{Overview of the Transport Layer in the Internet}
Op de transport layer heb je maar twee protocollen \textbf{UDP (User Datagram Protocol) en TCP (Transmission Control Protocol)}. UDP levert geen vertrouwlijke data overdracht. Wanneer dat niet aankomt word het niet gecontroleerd. TCP doet dat wel. UDP heeft daardoor wel een kleinere header en kan data versturen zonder een handshake waardoor het bij sommige gevallen sneller kan zijn.
\newline
IP/TCP word als een \textbf{best-effort delivery service en als unreliable service} gezien. Dit houd in dat het zijn best doet maar geen beloftes doet, je data kan dus niet aankomen of vermist raken. TCP bied wel \textbf{reliable data transfer} door te controleren of aan pakketje wel in aangekomen. TCP maakt ook gebruikt van flow control, acknowledgments en onderanderen \textbf{congestion control}. Congestion control is service die TCP bied aan het internet om er voor te zorgen dat haar pakketjes niet het hele internet overspoelen. Wanneer TCP merkt dat een node maar 10 pakketjes per seconde kan handelen en hij er zelf 100 per seconde stuurt, zal TCP dit terug schroeven om zo het netwerk niet onnodig te belasten.

\subsection{Multiplexing And Demultiplexing}
Een applicatie bind zich aan een socket, dit is waarmee hij communiseert met de transport laag en dus verbonden is met het internet.
\newline
Wanneer de transport laag een segment van de netwerk laag ontvangt inspecteerd hij deze en stuurt het door naar de bijbehoorende poort. Dit proces noemen we \textbf{demultiplexing}. Andersom van applicatie naar network heet \textbf{multiplexing}.
\newline
TCP en UDP hebben beide twee header fields gemeen: \textbf{source port, destination port}. Een port is 16bit en kan dus van 0 tm 65535 zijn. van 0 tot 1023 worden \textbf{well-known port numbers} gemoend. Deze zijn eigenelijk al bezet door ouroude internet applicaties.
TCP laat zichzelf identifiseren door middel van een four-tuple: source ip, source port, destination ip, dest. port.
\subsection{Connectionless Transport: UDP}
voordelen van UDP:
\paragraph{Finer application-level control over what data is sent, and when} Omdat de header van udp klein is stuur je geen dingen mee waarvan je niet weet wat het is. Omdat udp ook bijna geen controle uitvoert op het pakketje word het bijna rouw doorgestuurt naar de netwerk laag.
\paragraph{no connection establishment} Met UDP kan je meteen beginnen met versturen van data, het is niet nodig om eerst een \textbf{three-way hand-shake} te doen en/of om connecties open te houden. Wat bij TCP wel het geval is.
\paragraph{No connection state} UDP is stateless, je hoeft niet voor het verzenden connectie te maken.
\paragraph{Small packet header overhead} hier hebben we het al over gehad.
\subsubsection{UDP segment structure}
\subsubsection{UDP checksum}
udp heeft net als tcp ook checksums maar ze zijn een heel stuk toleranter. Wanneer udp merkt dat een pakketje niet klopt kan hij hem alsnog doorsturen (met flag) naar de applicatie laag.
\subsection{Principles of Reliable Data Transfer}
word niet behandelt. (pag 234)
\subsection{Connection-Oriented Transport TCP}
\subsubsection{The TCP Connection}
TCP is \textbf{connection-oriented} dat wil zeggen dat het eerst een actieve connectie moet hebben met een andere
host voordat het pakketjes kan ontvangen of versturen. Dit noemen we een hand-shake. De client vraagt aan de host of
hij verbinding mag maken, de host geeft antwoord dat hij het goed vind en de verbinding opent en de client reageert
weer dat hij het bericht van de host heeft ontvangen.
\newline
TCP is een \textbf{full-duplex service} dat houd in dat hij dingen kan ontvangen en versturen tegelijkertijd, in
tegenstelling to een walkie-talkie die maar half-duplex is, over. TCP is ook \textbf{point-to-point} dat wil zeggen
dat er maar 1 ontvangen kan zijn en maar 1 verzender.
\newline
Wanneer TCP data krijgt van de applicatie laag doet hij dat in een \textbf{send-buffer}. En van tijd tot tijd stuurt
hij dit door naar de netwerk-laag. De groote van de hoeveelheid dat hij pakt word bepaald door de \textbf{maximum
segment size (MSS)} voor de netwerk laag. Deze word weer bepaald door de \textbf{maximum transmission unit (MTU)} dit
is de maximale frame grote op de link laag. Dit is allemaal plus alle headers van de verschillende lagen en de data.

\subsubsection{TCP segment Structure}
tpc heeft net als udp \textbf{source en destination port numbers} in zijn header staan. Hij heeft ook een
\textbf{checksum}.
\newline
\begin{itemize}
    \item \textbf{sequence number field en acknowledgment numberfield} (beide 32bit) word gebruikt voor het verfiëren
    van het
    pakketje. Elk pakketje dat verstuurt word met TCP heeft een \textbf{ack} en \textbf{seq} nummer daarmaa kan de
    ontvanger controleren of hij alle pakketjes wel ontvangen heeft. Als een host een pakketje stuurt met seq: 1 van
    8 bytes, stuurt de ontvanger een pakketje terug met ack: 10 ($1 + 8 = 9$). Daarma wil hij zeggen dat hij alles
    tot 9 heeft ontvangen en klaar is voor pakketje 10.
    \item \textbf{receive window} (16bit) Dit veld word gebruikt voor flow-control. Hierin staat hoeveel bytes de
    ontvanger kan accepteren.
    \item \textbf{header length field} (4bit) In deze header staat hoegroot de header is.
    \item \textbf{options field} In dit veld onderhandelen de twee eind-systemen over de MSS.
    \item \textbf{flag field} Dit zijn allemaal mini flags van 6bit
        \begin{itemize}
            \item \textbf{ACK} de ACK waar we het eerder overhadden.
            \item \textbf{SYN & FIN} worden gebruikt voor het opzetten en afbreken van de connectie
            \item \textbf{PSH} als deze aan staat betekent het dat de ontvanger het bericht meteen moet pushen naar
            de applicatie laag
            \item \textbf{URG} geef aan dat de data "urgent" is. Dit wijst naar een \textbf{urgent data pointer
            field} (16bit), maar word nauwlijks gebruikt.
        \end{itemize}
\end{itemize}
De \textbf{sequance van een segment} is belangrijk om de goed volgorde aan te houden. Pakketjes hoeven niet altijd op
dezelfde volgorde te worden ontvangen.

\subsubsection{Round-Trip Time Estimation and Timeout}
word niet behandeld (pag 269)
\subsubsection{Reliable Data Transfer}
word niet behandeld (pag 272)
\subsubsection{Flow Control}
TCP bied \textbf{flow-control service}. Dit is een speed-matching systeem zodat de rate van data wat je verzend even
hoog is als de bottleneck van je netwerk. Dit kan op meerde methodes. Een van die methodes noem je \textbf{congestion
control} als je de snelheid op een "file" in het netwerk.
\newline
\textbf{receive window} word gebruikt door de verzender om te kijken hoeveel vrije buffer er nog is om dingen te
verzenden.


