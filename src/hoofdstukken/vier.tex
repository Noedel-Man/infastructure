\section{The Network Layer: Data Plane}
De netwerk laag kan je opdelen in twee delen: \textbf{data plane en control plane} In hoofdruk 4 gaat het meer over de data-plane.

\subsection{Overview of Network Layer}
\subsubsection{Forwarding and Routing: The Data and Control Planes}
Het heel doel van de netwerk laag is om data van een host naar de anderen te verplaatsen. Het bestaat uit
\begin{itemize}
    \item \textit{Forwarding} Het pakketje naar een de volgende host sturen.
    \item \textit{Routing} De route bepalen die het pakketje moet volgen. Dit doet hij met \textbf{routing algorithms}
\end{itemize}
Een router heeft een \textbf{forwarding table} waar instaat via welke interface pakketje naar buiten moeten op basis van hun bestemming.
\subsubsection{Network Service Model}
De enige service die de netwerk laag bied is \textbf{best effort}. Anderen services die \textbf{niet} toepassen op de netwerk laag maar wel bestaan:

\begin{itemize}
    \item \textit{Guaranteed delivery}
    \item \textit{Guaranteed delivery with bounded delay} zelfde als hierboven maar dan tijd gebonden
    \item \textit{In-order packet delivery}
    \item \textit{Guaranteed minimal bandwidth}
    \item \textit{Security}
\end{itemize}
\newline
Switches op de link-layer (\textbf{link-layer switches}) baseren hun beslissingen op de header in de link layer. Switches op de network-layer (\textbf{routers}) doen die beslissingen op basis van de network-layer header.

\subsection{What's Inside a Router?}
Het router gedeelte van een router bestaat uit de volgende dingen:
\begin{itemize}
    \item \textit{Input ports} Dit is waar de pakketjes binnen komen. word uitgepakt tot datagram . Dan wordt gekeken waar de het pakketje naartoe moet en word dan (als hij de voorste in de queue is) Doorgestuurt naar de \textbf{Switching fabric}
    \item \textit{Switching fabric} hier kunnen maar een aantal pakketjes tegelijkertijd opzitten (meer hierover later) maar  de switching fabric stuurt ze naar de goede \textbf{output ports}
    \item \textit{Output ports} Hier worden de pakketjes weer ingepakt naar de link-layer en dan physical layer en verstuurt over het internet.
    \item \textit{Routing processor} Dit ding zorgt ervoor dat het pakketje doe goede richting opgaat. Bij ISP hebben ze zelfs een \textbf{software definded routing} (SDN?) zodat het vanuit een centrale plek gebruikt. Traditioneel gezien delen routers hun forwarding table met elkaar en ze weten ze waar hun pakketje naartoe moet.
\end{itemize}
\newline
De dataplane doet dingen op de nano-seconde en zit ook vaak in hardware omdat het zo snel moet zijn (forwarding etc.) Dingen in de \textbf{control plane} werken vaak op een normale CPU met software.
\newline
Er zijn twee verschillende manieren van forwarden in een router:
\begin{itemize}
    \item \textit{Destination-based forwarding} een pakketje naar een link sturen omdat hij dan zo snel mogelijk op zijn bestemming komt.
    \item \textit{Generalized forwarding} Hij kan een pakketje ook naar een link forwarden op basis van zijn afkomst, inhoud of omdat hij niet weet waar hij naar toe moet gaan.
\end{itemize}
\subsubsection{Input Port Processing and Destination-Based Forwarding}
%Het forwarden van een pakketje op basis van zijn bestemming word ge
\subsubsection{Switching}
De \textbf{switching fabric} kan op drie verschillende manier werken:
\begin{itemize}
    \item \textit{Switching via memory} Dit is de simpelste methode, de data van de input port word gekopieerd naar de ram van de router en gekopieerd naar de output. Met deze methode kan je maar een pakketje tegelijkertijd processen.
    \item \textit{Switching via a bus} Met deze methode word het pakketje naar alle outputs tegelijk verstuurt via een bus, maar met een label voor de goede output. De anderen outputs negeren het bericht. Ook maar een bericht tegelijkertijd.
    \item \textit{Switching via an interconnection network (cross-bar)} met deze methode kan je meerde pakketjes tegelijktijd versturen zolang ze maar niet naar de zelfde input en/of output gaan. een pakketje van input A kan naar output X terwijl er ook een pakketje van input B naar output Y gaat. Cross bar is \textbf{non-blocking}
\end{itemize}
\subsubsection{Output Port Processing}

\subsubsection{Where Does Queuing Occur}
het queue van een router gebeurt na de route bepaling in de input ports en aan het gebin van de output ports. Wanneer die queue vol zit kan er \textbf{packet loss} voor komen als er nog meer pakketjes bijkomen. Die worden dan gedropt.


\subsection{The Internet Protocol (IP): IPv4, Addressing, IPv6 and More}
 Er bestaan anno 2017 twee versies van IP (internet protocol): Ipv4 en Ipv6. IPv4 word het meest gebruikt maar heeft maar een hele kleine hoeveelheid adressen 32bit. Die van IPv6 zijn 128 bits.

\subsubsection{IPv4 Datagram Format}
Het IPv4 datagram bestaat uit de volgende onderdelen:
\begin{itemize}
    \item \textit{Version number} 4bits, staat het versie nummer van de IP.
    \item \textit{Header lenght} 4 bits, beschrijft hoe groot de header is. Meestal 20 bytes.
    \item \textit{Type of Service} (TOS) hier wordt beschreven wat voor soort data dit is. realtime, VOIP etc.
    \item \textit{Datagram length} 16 bit, dit is de totale lengte van de datagram (header+data) meestal niet meer dan 1,500 bytes (max 65,535)
    \item \textit{Identifier, flags, fragmentation offset} IPv6 heeft geen fragmentation
    \item \textit{Time-to-live (TTL)} 8bit, In dit veld word aangegeven hoeveel hops dit datagram nog mag maken. Wanneer hij bij een nieuwe router komt word deze waarde $-1$ gedaan. Wanneer het $0$ berijkt moet de router hem droppen.
    \item \textit{Protocol} 8bit, Hierin staat welke transport-layer protocol deze datagram naar moet wordengepaast wanneer hij op zijn bestemming is. Dus TCP of UDP
    \item \textit{header checksum} 16bit, checksum voor de header.
    \item \textit{Source and destination IP addresses} 32bit p.s, hier staan de source en destination ip in allebij in een apart veld van 32bit.
    \item \textit{Options} Hier kunnen not extra dingen staan, word bijna niet gebruikt. IPv6 heeft dit veld niet meer.
    \item \textit{Data} De data van de datagram. (TCP, UDP, ICMP etc.)
\end{itemize}

\subsubsection{IPv4 Datagram Fragmentation}
Een Ethernet frame heeft een \textbf{maximum transmisson unit (MTU)} van 1,500 bytes. Maar somige wide-area link op
het internet kunnen hebben maar een MTU van 576 bytes. Als dat gebeurt word het pakketje opgesplits in het maximale
MTU van de bottleneck. Dit noem je een \textbf{fragment}. Het pakketje word daarna \textbf{niet} weer opnieuw groter
gemaakt als hij de bottleneck uit is. Bij IPv6 stuurt de router een ICMP bericht naar de verzender om te zeggen dat
hij een kleinere MTU moet doen. Dit hebben ze veranderd omdat het voor de router te process intensief is.

\subsubsection{IPv4 Addressing}
De connectie tussen een host en een link noem je een \textbf{interface}.
\newline
IPv4 addresen zijn 32bit, er kunnen dus 4 miljard unieke addresen bestaan. IPv4 word eigenlijk altijd opgeschreven in
een \textbf{dotted-decimal notation} (193.32.216.9) elk getal tussen de puntjes is 8 bits, je zou dit ook in binaur
kunnen opschrijven
$$11000001 00100000 11011000 00001001$$
Je kan in een netwerk \textbf{subnets} maken, subnets kan je onderverdelen per interface van je router(s). Zodat je
je netwerk gescheiden houd. Als je dan een ip-range aan een bepaalde subnet wilt toekennen heb je te maken met een
\textbf{subnet mask}, daarmee vertel je hoeveel bits je toekent dus \textit{192.168.0.0/28} ken je de laatste 4 bit toe
aan je subnet ($32-28=4$). Wat neerkomt op 16 addresen, daarvan zijn er altijd 2 gereserveerd. een voor je default
gateway (router) en een voor de broadcast (voor DHCP bijvoorbeeld). Je houd er dus 14 over om je hosts op aan te
sluiten.
\newline
De internet addres toekennings methode heet: \textbf{Classless interdomain Routing (CIDR)} van oorspring had je
namelijk subnet classes. Waavan 10.0.0.0/8 classe A was, 10.0.0.0/16 B, 10.0.0.0/24 C. Maar met deze methode heb je
of 255 (C) addresen of 65536 (B) of 16777216 (A) Terwijl je er als organistie maar vaak 1k nodig had.
\newline
ISP krijgen een blok aan ipaddresen en ddres staat niet vast en je kan dus altijd een andere krijgen.
\newlinedelen deze ook weer uit aan hun klanten. Van een locale ISP zoals Ziggo krijg
je een \textbf{dynamic ip addres} dit a
Wanneer jij verbinding maakt met een nieuw netwerk heb je een IP addres nodig. Die kan je krijgen van een
\textbf{Dynamic Host Configuration Protocol (DHCP)} server, ieder netwerk heeft een of meerdere van dit soort servers
. Je krijgt van een DHCP server een \textbf{temporary IP address} Deze heeft een lease-tijd en moet je om een
bepaalde tijd weer vernieuwen bij je DHCP server. gelukkig gebeuren al deze dingen automatish want DHCP is een
\textbf{zero-conf/plug-and-play} systeem.
\newline
Wanneer een hosts een ip-addres wilt moet hij doro een vier-stappen plan heen van DHCP, het zogenaamde \textbf{DORA}
\begin{itemize}
    \item \textit{DHCP server discovery} stap 1 is om de DHCP server te vinden. Dit doet een host door middel van
een \textbf{DHCP discovery message} dit word naar ip 255.255.255.255 gestuurd, het broadcast IP. Op 255.. staat geen
host maar is bedoelt als broadcast, iedereen in dezelfde subnet ontvangt dit bericht. In dit bericht staat zijn ID van zijn request samen met nog wat info.
    \item \textit{DHCP server offer(s)} De DHCP server ontvangt het bericht en stuurt een broadcast terug het netwerk
    op met een \textbf{DHCP offer message}. Hij doet een broadcast omdat de andere host nog geen IP addres heeft. In
    dit offer staat een IP addres die hij mag gebruiken. Samen met een \textbf{lease-time} (hoelang hij dit IP addres
    mag gebruiken)
    \item \textit{DHCP request} De client stuurt een \textbf{DHCP request message} met het IP dat hij wilt hebben (de
    host kan meerde offers gehad hebben van verschillende DHCP servers)
    \item \textit{DHCP ACK} de DHCP server stuurt een \textbf{DHCP ACK message} dat alles OK is.
\end{itemize}

\subsubsection{Network Address Translation (NAT)}
Omdat er maar 4 miljard IP addresen zijn en toch iedereen met zijn, computer, laptop, tablet en telefoon
tegelijkertijd op het internet wilt gaan maken we gebruikt van \textbf{Network address translation (NAT)}. NATs
worden toegepast op \textbf{private networks}. Je krijgt van Ziggo, KPN, xs4alll of elke andre ISP maar 1 IPv4 (soms
ook een IPv6) addres. Maar je hebt ondertussen wel 20 hosts thuis staan die allemaal verbonden moeten worden. Je
router maakt dan gebruikt van NAT, hij kens alle host binnen je netwek een ip addres toe van 192.168.0.0/16 of 10.0.0
.0/8 dit zijn gereserveerde adderesen voor NATs. Er bestaat geen public adderesen met deze waardes. Ze bestaat dus
alleen in je thuis netwerk. Als je een pakketje verstuurt naar een host buiten je thuis netwerk pikt je router dit op
en veranderd de source ip en de source port naar het \textbf{public IP} en een andere port. Wat je van je ISP hebt
gekregen. Hij slaat alle connecties op in een \textbf{NAT translation table} (source-dest ip, port) zodat hij weet welk binnen komend paktje voor welk private ip zijn. Hij vervangt dan de port nr en dest ip en stuurt het pakketje weer
door.
\newline
Wanneer je deel wilt nemen in een p2p (peer-to-peer) connectie of een ander soort verbinding waar het belangrijk is
dat je poort nummer kloppen kan een host gebruik maken van \textbf{NAT traversal} tools en UPnP die configureren je
NAT in je router dat hij niet aan de poort nummers mag komen. En dat hij alles van port \textit{x} automatich forward
naar een specivike host.
Een NAT valt onder de categorie \textbf{middlebox}

\subsubsection{IPv6}
De belangrijkste veranderingen in IPv6 zijn:
\begin{itemize}
    \item \textit{Expanded addressing capabilities} ipv6 is 128bit inplaats van 32bit van v4. IPv6 ondersteund ook
    \textbf{anycast address} hiermee kan je een broadcast doen naar een groep public ip addresen.
    \item \textit{A streamlined 40-byte header} ipv6 heeft een fixed 40 byte header
    \item \textit{flow labeling} Je kan aangeven wat voor service je datagram is. (real-time, high priority etc)
\end{itemize}
Een Ipv6 datagram bestaat uit de volgende onderdelen:
\begin{itemize}
    \item \textit{Version} 4bit, de IP versie.
    \item \textit{Traffic class} 8bit, hier kan je een hogere priority geven aan je datagram als het real-time data
    \item \textit{Flow label} 20bit, zoals beschreven in flow labeling
    \item \textit{payload length} 16bit, hoeveel bytes de header+data is
    \item \textit{Next header} welke transport layer protocol word gebruikt, zelfde als IPv4
    \item \textit{Hop limit} zelfde las TTL in IPv4
    \item \textit{source and desination addresses} 128bit p.s. source en destination in een apart veld.
    \item \textit{Data} de data.
\end{itemize}
IPv4 vs IPv6:
\begin{itemize}
    \item \textit{fragmentation/reassembly} v6 bied geen fragmentation. Als de MTU bij een link ineens kleiner moet
    word er een ICMP bericht gestuurt naar de verzender van het pakketje dat hij het opnieuw moet sturen met een
    kleinere MTU
    \item \textit{Header checksum} v6 heeft geen checksum omdat het in de andere layers al meer dan genoeg voorkomen.
    \item \textit{Options} v6 heeft geen opties.
\end{itemize}
Wanneer je een IPv6 pakketje over het internet verstuurt is de kans groot dat hij bij een router komt die IPv6 niet
ondersteund. De laatste router die wel v6 ondersteund stopt dan de gehele v6 datagram in een IPv4 datagram en die
adraseert hij naar de eerste node op de route die wel v6 ondersteund. Deze router leest het pakketje dan, pakt het
uit en verstuurt het orginele v6 pakketje alsof er niks gebeurt is. Dit process noemen we \textbf{tunneling}.