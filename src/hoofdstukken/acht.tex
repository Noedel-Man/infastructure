\addtocounter{section}{1}
\section{Security in Computer Networks}
\subsection{What is Network Security?}
Een veilige verbinden kan bestaan uit de volgende eigenschappen van \textbf{secure communication}
\begin{itemize}
    \item \textit{Confidentiality} Het bericht kan alleen worden gelezen door de ontvangen er en de verzender. Dit kan je krijgen door gebruikt te maken van \textbf{encrypted}
    \item \textit{Message intergrity} Je wilt er zeker van zijn dat het bericht wat je ontvangt ook precies hetzelfde is wat jij hebt verstuurt.
    \item \textit{End-point authentication} Je wilt er zeker van zijn dat de ontvanger ook echt is wie hij zecht dat hij is.
    \item \textit{Operational security} Je wilt dat je netwerk veilig is om op te werken.
\end{itemize}
\subsection{Principles of Cryptography}
Waarneer je blanko tekst versuurt staat dat bekent als \textbf{cleartext/cleartext} je kan met een \textbf{encryption algorithm} je data encrypten, dan staat het bekent als \textbf{ciphertext}
\newline
Om data geencrypt te versturen maak je gebruik van \textbf{keys} Als je data encrypt met een key, de data verstuurt naar een ontvanger en de ontvanger gebruikt dezelfde key om de data te decrypte noem je dit \textbf{symmetric key system}. Het probleem hiervan is dat je eerst een key moet afspreken.
\newline
Met \textbf{public key system} heeft iedere host 2 keys, een public en een private. Je private key deel je met
niemand. Deze gebruik je om berichte te decrypten (of encrypten maar dan komt later). Je public key iedereen bij.
Iemand kan dus een bericht encrypten met jou public key die kan dan alleen worden gedecrypt met jou private key.

\subsubsection{Symmetric Key Cryptography}
\textbf{Monoalphabetic cipher} is een encryptie dat je een letter door een andere letter vervangt.
drie verschillende aanvallen op encryptie:
\begin{itemize}
    \item \textbf{Ciphertext-only attack} Met deze aanval heb je toegang tot de ciphertext zonder identicatie wat het
    bericht zecht.
    \item \textbf{Known-plaintext attack} Nu heb je de cipertext en weet je (gedeeltelijk) wat er in het bericht
    staat, makkelijker te decrypten dus.
    \item \textbf{Chosen-plaintext attack} Nu kan de aanvaller text incrypten opdezelfde methode als het
    oorspronkelijke bericht.
\end{itemize}
\textbf{Polyalpabetic encryption} is dat je op basis van de positie van de zin de letter een andere letter maakt.

\subsubsection{Public Key Encryption}
Met de public/private key methode heeft iedereen twee keys, eentje kan iedereen zien: de public key. En de anderen
niemand, de private key. Wanneer je een bericht encrypt met iemand zijn public key kan deze alleen maar worden ge
degript mijn zijn private key. Anderszom werkt dat dus ook. Zo kan je zeker weten dat een bericht van iemand afkomt
als hij het met zijn private key heeft geencrypt (\textbf{end-point authentication}). En als je een bericht met iemand
z'n public key encrypt berijk je \textbf{confidentiality}. Omdat je alleen het bericht kan lezen met zijn private key.

\subsection{Message Intergrity and Digital Signatures}
Met de public key methode kan je ervoor zorgen dat je confidentiality en end-point authentication krijgt. We missen
echter nog \textbf{message integrity}.

\subsubsection{Cryptographic Hash Functions}
Met hash functions kan je je bericht hashen en dan het bericht samen met de hash meesturen. De ontvangen voert dan
dezelfde hash functie uit op het bericht en vergelijkt die dan met de meegestuurde hash.

\subsubsection{Message Authentication Code}
Om ervoor te zorgen dat niemand de hash kan namaken moet je gebruik maken van een \textbf{Authentication key} dit is
een kleine string aan bits die je bij je bericht doet en daar vervolgends een hash van maakt. Anderen mensen weten
die \textit{secret} code niet.
\newline
Een \textbf{message authentication code (MAC)} is het bericht + secret in een hash (\textit{H(b+s)}). Er zijn
verschillende standaarden voor bijvoorbeeld \textbf{HMAC} wat MD5 of SHA-1 gebruikt.

\subsubsection{Digital Signatures}
Als je een bericht encrypt met je private-key heeft dat bericht dus een \textbf{digital siganture} omdat je kan
controleren wie het heeft geschreven door het bericht te decrypten met je public key.
\newline
Om te controleren of een public wel echt van een persoon A is en niet van iemand anders bestaat er \textbf{public key
certification}. Dat doe je bij een \textbf{Certification Authority (CA)} deze houd bij welke persoon/organisatie bij
welke key hoort.

\addtocounter{subsection}{2}
\subsection{Securing TCP Connections: SSL}
Omdat er geen Security layer bestaat in het IP/TCP model, maar hier wel veel vraag naar is, is er een soort van
virtuele laag gemaakt op de applicatie layer voor tcp. Dit staat bekent als \textbf{Secure Sockets Layers (SSL)} en
bied services als confidentiality, Message intergrity en end-point authentication. Een aangepast versie van SSL v3
heet \textbf{Transport Layer Security (TLS)}

\subsection{Network-Layer Security: IPsec and Virtual Private Networks}
De IP security protocol beterbekent als \textbf{IPsec} is een beveiligings protocol op de netwerk laag. Al je
datagrammen zijn dus beschermt als je IPsec gebruikt. IPsec word ook gebruikt om \textbf{virtual private networks
(VPNs)} te maken.
\subsubsection{IPsec and Virtual Private Networks (VPNs)}
IPsec gaat pas in als het langs een router naar de buiten wereld gaat. In je interne netwerk zijn je datagrammen dus
niet encrypted. Je router moet ook IPsec ondersteunen anders werkt het zoiezo niet.
\newline
Wanneer je een \textbf{private nerwork} wilt opzetten met iemand die buiten je LAN zit gebruik je ook IPsec. Ookal
kan de router van de andere persoon er evniks mee. Het word dan gewoon door gestuurt naar zijn laptop en die decrypt
het IPsec datagram met software op de applicatie laag.

\addtocounter{subsection}{1}
\subsection{Operational Security: Firewalls and Intrusion Detection Systems}
\subsubsection{Firewalls}
een \textbf{firewall} is een stuk hardware en software wat je interne prive netwerk isoleert van de buiten wereld.
De drie doelen van een firewall zijn:
\begin{enumerate}
    \item \textit{All traffic from outside to inside, and vice versa, passes through the firewall}
    \item \textit{Only authorized traffic, as defined bt the local security policy, will be allowed to pass}
    \item \textit{The firewall itself is immume to penetration}
\end{enumerate}
Firewall kan je onderverdelen in drie categorieën:
\begin{itemize}
    \item \textbf{Traditional packet filters} iedere pakket die het netwerk binnen komt of verlaat word gescreened
    op: IP source-dest, protocol: TCP, UDP, ICMP, TCP ACK-SYN flag, etc. Op basis daarvan kan de firewall beslissen om
    pakketjes door te laten.
    \item \textbf{stateful packet filters} wanneer je een nieuwe verbinding maakt word dat opgeslagen in je de
    connection table van je firewall. Zodat hij kan zien of een binnen komende connectie van een openstaande
    verbinding is of dat hij ineens een host binnen het netwerk wilt bereiken die eigenlijk helemaal geen server is.
    \item \textbf{Application Gateway} Deze kijkt naar de data van het pakketje en maakt op basis daarvan
    beslissingen. Zo kan hij services als TELNET toestaan voor bepaalde gebruikers (ip addressen).
\end{itemize}

\subsubsection{Intrusion Detection Systems}
Een firewall is niet altijd genoeg om kwaadaardige pakketjes tegen te houden. Daarvoor kijkt de firewall niet (lang)
genoeg naar de detail van de data van het pakketje. Daarvoor hebben we \textbf{deep packet inspection}. Daarvan heb
je twee smaken, bijde werken op application-layer niveau en staan los van de router/firewall.
\begin{itemize}
    \item \textbf{intrusion detection system (IDS)} Deze inspecteerd alle pakketjes op het netwerk en geeft een
    signaal als het iets raars tegenkomt.
    \item \textbf{intrusion prevention system (IPS)} De inspecteerd en blokeerd als hij iets abnormaals tegenkomt.
\end{itemize}
Een IDS/IPS kan op twee verschillende manieren werken:
\begin{itemize}
    \item \textbf{signature-based system} De IDS/IPS inspecteerd pakketje op basis van bekende vijandige pakketjes
    signatures, hij kijkt dan naar: protocol, data, poort, groote etc. En vergelijkt dat met een database.
    \item \textbf{anomaly-based system} Met deze methode kijkt de IDS/IPS naar het netwerk of er iets ongebruikelijks
    plaats vind. Bijvoorbeeld dat er ineens om 3 uur s'nachts een FTP verbinding word geopent waar 10gb aan data
    doorheen gaat. Of dat een host die normaal alleen e-mails verstuurt nu ineens een remote-desktop verbinden buiten
    het netwerk heeft opgezet.
\end{itemize}
Snort is een IDS/IPS programma. Je wilt er vaak meerdere in je netwerk hebben op verschillende punten om zo ook
interne communicatie te monitoren.
\newline
Een \textbf{demilitarized zone (DMZ)} is een gebied in je netwerk die afgeschermt is van de rest van je netwerk. Dit
word vaak gebruikt om servers te hosten. Daar gebruik je dan een aparte firwall en/of router voor. Je kan dan
makkelijk anderen regels per gebied opzetten. En als een hacker in de DMZ komt kan hij niet meteen bij de rest van
het netwerk en visaversa.